\documentclass[11pt]{article}
\usepackage[english]{babel}
\usepackage{url}
\usepackage{graphicx,DCCN2020_en}

\pagestyle{fancy}
\fancyhead{} 
\fancyfoot{}

\usepackage[utf8]{inputenc}
\linespread{1.0}

\usepackage{amsmath}

\makeatletter
\fancyhead[RO]{\small DCCN 2020\\ {14-18 September 2020}}
\fancyhead[LO]{\small P.D. Petrov, G.B. Kostadinov, et al.\\
Approximate Sequencing of Virtual Reels with GA}

\c@page=1
       
\makeatother

\title{Approximate Sequencing of Virtual Reels with Genetic Algorithms}

\author[]{\small P.D. Petrov}
\author[]{\small G.B. Kostadinov}
\author[]{\small P.R. Zhivkov}
\author[]{\small \\V.I. Velichkova}
\author[]{\small T.D. Balabanov\textsuperscript{0000-0003-3139-069X}}
\affil[]{\footnotesize Bulgarian Academy of Sciences \\ Institute of Information and Communication Technologies \\ acad. Georgi Bonchev Str., block 2, office 514 \\ 1113 Sofia, Bulgaria \\ http://iict.bas.bg/}
\email{p.petrov@iit.bas.bg g.kostadinov@iit.bas.bg pzhivkov@iit.bas.bg vvelichkova@iit.bas.bg todorb@iinf.bas.bg}
%
% Plamen Dimitrov Petrov - p.petrov@iit.bas.bg
% Georgi Borisov Kostadinov - g.kostadinov@iit.bas.bg
% Petar Rumenov Zhivkov - pzhivkov@iit.bas.bg
% Veneta Ivanova Velichkova - vvelichkova@iit.bas.bg
% Todor Dimitrov Balabanov - todorb@iinf.bas.bg

\begin{document}

\udc{519.2}

{\let\newpage\relax\maketitle}

\vskip -1.5em

\footnotetext{The publication has been prepared with the support of Velbazhd Software LLC and Bulgarian Ministry of Education and Science according to the research project No.{D01–205/23.11.2018}.}

\begin{abstract}
Sequencing is a very popular mathematical problem in the field of genetics. DNA sequence information is organized as pairs of the four nucleotide bases - Cytosine, Guanine, Adenine, and Thymine. In some cases, only chunks are known but the full sequence is unknown. The problem of sequencing is a reconstruction of the full sequence from the known chunks. Sequencing is applied also in other fields as encoding and cryptography. This research proposes approximate sequencing of virtual reels used in gambling slot machine games. The optimization process is done with classical genetic algorithms, but optimality is estimated into chunks space instead of sequences space.
\keywords{Sequencing, Slot Machines, Genetic Algorithms}
\end{abstract}

\section{Introduction}

\section{Conclusion}

\section*{Acknowledgments}

This research is funded by Velbazhd Software LLC and it is partially supported by the Bulgarian Ministry of Education and Science (contract D01–205/23.11.2018) under the National Scientific Program ``Information and Communication Technologies for a Single Digital Market in Science, Education and Security (ICTinSES)'', approved by DCM \# 577/17.08.2018.

\begin{thebibliography}{99}

\bibitem{}

\end{thebibliography}

\end{document}
